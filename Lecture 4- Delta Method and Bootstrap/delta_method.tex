\documentclass[handout, aspectratio=169, 10pt]{beamer}

% packages
\usepackage{newpxmath} % math font is Palatino compatible
% \usepackage[nomath]{fontspec}

\usepackage{setspace}
\usepackage{xcolor}
\usepackage{soul} % for \st
\usepackage{hyperref} % for links
\definecolor{links}{HTML}{2A1B81}
\hypersetup{colorlinks,linkcolor=,urlcolor=links}


% table stuff
\usepackage{chronosys}
\usepackage{verbatim}
% \pagenumbering{arabic}
\usepackage{tabularx}
\usepackage{booktabs}
\usepackage{ragged2e}
\usepackage{mathtools}

% R Code
\usepackage{listings}
\usepackage{courier}
\lstset{basicstyle=\scriptsize\ttfamily,breaklines=true}
\lstset{framextopmargin=50pt,frame=bottomline}

% themes
\usetheme[progressbar=frametitle, block=fill]{metropolis}
\useoutertheme{metropolis}
\useinnertheme{metropolis}

% colors
\definecolor{dimwhite}{rgb}{0.99, 0.99, 0.99}
\definecolor{charcoal}{rgb}{0.21, 0.27, 0.31}
\definecolor{slategray}{rgb}{0.44, 0.5, 0.56}
\definecolor{dimgray}{rgb}{0.41, 0.41, 0.41}
\definecolor{bleudefrance}{rgb}{0.19, 0.55, 0.91}

% beamer options
\setbeamercolor{author}{fg=charcoal}
\setbeamercolor{background canvas}{bg=white}
\setbeamercolor{section in toc}{fg=charcoal}
\setbeamercolor{subsection in toc}{fg=dimgray}
\setbeamercolor{frametitle}{bg=dimwhite, fg=charcoal}
\setbeamercolor{progress bar}{fg=slategray, bg=fg!50!black!30}
\setbeamercovered{transparent}
\setbeamertemplate{itemize items}[triangle]
\setbeamertemplate{itemize subitem}[circle]
\setbeamertemplate{itemize subsubitem}[square]
\setbeamersize{text margin left=7mm,text margin right=7mm} 

% new commands
\newcommand{\q}[1]{``#1''}
\newcommand{\hs}[1]{\textsc{\hfill\scriptsize\color{dimgray}#1}}
\newcommand{\g}[1]{{\color{gray}#1}}
\newcommand{\dg}[1]{{\color{dimgray}#1}}
\newcommand{\sg}[1]{{\color{slategray}#1}}
\newcommand{\bdf}[1]{{\color{bleudefrance}#1}}
\newcommand{\itemcolor}[1]{\renewcommand{\makelabel}[1]{\color{#1}\hfil ##1}}
\newcommand\Wider[2][2em]{
\makebox[\linewidth][c]{
  \begin{minipage}{\dimexpr\textwidth+#1\relax}
  \raggedright#2
  \end{minipage}
  }
}

% misc
\linespread{1.35}

% Math stuff
\newcommand{\norm}[1]{\left\lVert#1\right\rVert}
\newcommand{\R}{\mathbb{R}}
\newcommand{\E}{\mathbb{E}}
\newcommand{\V}{\mathbb{V}}
\newcommand{\probP}{\text{I\kern-0.15em P}}
\newcommand{\ol}{\overline}
%\newcommand{\ul}{\underline}
\newcommand{\pp}{{\prime \prime}}
\newcommand{\ppp}{{\prime \prime \prime}}
\newcommand{\policy}{\gamma}
\newcommand{\plim}{ \overset{p}{\to}}
\newcommand{\hnot}{ \overset{H_0}{\to}}

% Causal Graphs
\usetikzlibrary{shapes,decorations,arrows,calc,arrows.meta,fit,positioning}
\tikzset{
    -Latex,auto,node distance =1 cm and 1 cm,semithick,
    state/.style ={ellipse, draw, minimum width = 0.7 cm},
    point/.style = {circle, draw, inner sep=0.04cm,fill,node contents={}},
    bidirected/.style={Latex-Latex,dashed},
    el/.style = {inner sep=2pt, align=left, sloped}
}

\title [Bootstrap]{Delta Method}
\author{C.Conlon}
\institute{Applied Econometrics II}
\date{\today}
\setbeamerfont{equation}{size=\tiny}
\begin{document}

\begin{frame}
\titlepage
\end{frame}


\begin{frame}{Bootstrap and Delta Method}
\begin{itemize}
\item We know how to construct confidence intervals for parameter estimates:  $\hat{\theta}_k \pm 1.96 SE(\hat{\theta}_k)$
\item Often we are asked to construct standard errors or confidence intervals around model outputs that are not just parameter estimates: ie:  $h(x_i,\hat{\theta})$.
\item Sometimes we can't even write $g(x_i,\theta)$ as an explicit function of $\theta$ ie: $\Psi(h(x_i,\theta),\theta) = 0$.
\item Two options:
\begin{enumerate}
\item Delta Method
\item Bootstrap
\end{enumerate}
\end{itemize}
\end{frame}

\begin{frame}{Delta Method}
Delta method works by considering a \alert{Taylor Expansion} of $g(x_i,\theta)$.
\begin{eqnarray*}
h(z) \approx h(z_0) + h'(z_0)(z-z_0) + o(||z-z_0||)
\end{eqnarray*}
Assume that $\theta_n$ is asymptotically normally distributed so that:
\begin{eqnarray*}
\sqrt{n} (\theta_n - \theta_0) \sim N(0,\Sigma)
\end{eqnarray*}
(How do we get this: OLS? GMM? MLE?). Then we have that 
\begin{eqnarray*}
\sqrt{n} (h(\theta_n) - h(\theta_0)) \sim N(0,D(\theta)' \Sigma  D(\theta))
\end{eqnarray*}
Where $D(\theta) = \frac{\partial h(x_i, \theta)}{\partial \theta}$ is the Jacobian of $g$ with respect to theta evaluated at $\theta$.\\
We need $g$ to be continuously differentiable around the center of our expansion $\theta$.
\end{frame}


\begin{frame}{Delta Method: Examples}
Start with something simple: $g(\theta)= \overline{X}_1\cdot \overline{X}_2$ with $(X_{1i},X_{2i}) \sim IID$.
We know the CLT applies so that:
\begin{eqnarray*}
\sqrt{n}
\begin{pmatrix}
\overline{X}_1 - \mu_1\\
\overline{X}_2 - \mu_2
\end{pmatrix} &\sim  N
\begin{bmatrix}
\begin{pmatrix}
0\\
0
\end{pmatrix},
\Sigma
\end{bmatrix}
\end{eqnarray*}
The Jacobian is just $D(\theta) =  \begin{pmatrix}\frac{\partial g(\theta)}{\partial \theta_1} \\ \frac{\partial g(\theta)}{\partial \theta_2}  \end{pmatrix} =  \begin{pmatrix}s_2\\ s_1 \end{pmatrix}$\\
So,
\begin{eqnarray*}
V(Y) = D(\theta)' \Sigma D(\theta) =\begin{pmatrix} \mu_2 & \mu_1 \end{pmatrix}  \begin{pmatrix} \sigma_{11} & \sigma_{12} \\ \sigma_{21} & \sigma_{22} \end{pmatrix} \begin{pmatrix} \mu_2 \\\mu_1  
\end{pmatrix} \\
\sqrt{n} ( \overline{X}_1 \overline{X}_2 - \mu_1 \mu_2) \sim N(0,\mu_2^2 \sigma_{11}^2 + 2 \mu_1 \mu_2 \sigma_{12}  + \mu_1^2 \sigma_{22}^2)
\end{eqnarray*}
\end{frame}

\begin{frame}{Delta Method: Examples}
Think about a simple logit:
\begin{eqnarray*}
\probP(Y_i=1 | X_i ) = \frac{\exp^{\beta_0 + \beta_1 X_i}}{1+\exp^{\beta_0 + \beta_1 X_i}}  \quad \probP(Y_i=0 | X_i ) = \frac{1}{1+\exp^{\beta_0 + \beta_1 X_i}} 
\end{eqnarray*}
Remember the ``trick'' to use GLM (log-odds):
\begin{eqnarray*}
\log \probP(Y_i=1 | X_i) - \log \probP(Y_i=0 | X_i) = \beta_0 + \beta_1 X_i
\end{eqnarray*}
\begin{itemize}
\item Suppose that we have estimated $\hat{\beta_0},\hat{\beta_1}$ via GLM/MLE but we want to know the confidence interval for the probability: $P(Y_i=1 | X_i,\hat{\theta})$
\item The derivatives are a little bit tricky, but the idea is the same.
\item This is what STATA should be doing when you type: \tt{mfx, compute}
\end{itemize}
\end{frame}

\begin{frame}{Delta Method: Other Examples}
Often we have a regression like:
\begin{eqnarray*}
log Y_i = \beta_0 + \beta_1 X_i + \gamma Income_i + \epsilon_i
\end{eqnarray*}
And we are interested in $\beta_1 / \gamma$ so that we have $\beta_i$ in units of ``dollars''. Again Delta Method Works fine here.\\
\end{frame}


\begin{frame}{Delta Method: Some Failures}
But we need to be careful.  Suppose that $\theta \approx 0$ and 
\begin{itemize}
\item $h(x)  = |X|$
\item $h(x)  = 1/X$
\item $h(x)  = \sqrt{X}$
\end{itemize}
These situations can arise in practice when we have weak instruments or other problems.
\end{frame}



\end{document}