\documentclass[handout, aspectratio=169, 10pt]{beamer}

% packages
\usepackage{newpxmath} % math font is Palatino compatible
% \usepackage[nomath]{fontspec}

\usepackage{setspace}
\usepackage{xcolor}
\usepackage{soul} % for \st
\usepackage{hyperref} % for links
\definecolor{links}{HTML}{2A1B81}
\hypersetup{colorlinks,linkcolor=,urlcolor=links}


% table stuff
\usepackage{chronosys}
\usepackage{verbatim}
% \pagenumbering{arabic}
\usepackage{tabularx}
\usepackage{booktabs}
\usepackage{ragged2e}
\usepackage{mathtools}

% R Code
\usepackage{listings}
\usepackage{courier}
\lstset{basicstyle=\scriptsize\ttfamily,breaklines=true}
\lstset{framextopmargin=50pt,frame=bottomline}

% themes
\usetheme[progressbar=frametitle, block=fill]{metropolis}
\useoutertheme{metropolis}
\useinnertheme{metropolis}

% colors
\definecolor{dimwhite}{rgb}{0.99, 0.99, 0.99}
\definecolor{charcoal}{rgb}{0.21, 0.27, 0.31}
\definecolor{slategray}{rgb}{0.44, 0.5, 0.56}
\definecolor{dimgray}{rgb}{0.41, 0.41, 0.41}
\definecolor{bleudefrance}{rgb}{0.19, 0.55, 0.91}

% beamer options
\setbeamercolor{author}{fg=charcoal}
\setbeamercolor{background canvas}{bg=white}
\setbeamercolor{section in toc}{fg=charcoal}
\setbeamercolor{subsection in toc}{fg=dimgray}
\setbeamercolor{frametitle}{bg=dimwhite, fg=charcoal}
\setbeamercolor{progress bar}{fg=slategray, bg=fg!50!black!30}
\setbeamercovered{transparent}
\setbeamertemplate{itemize items}[triangle]
\setbeamertemplate{itemize subitem}[circle]
\setbeamertemplate{itemize subsubitem}[square]
\setbeamersize{text margin left=7mm,text margin right=7mm} 

% new commands
\newcommand{\q}[1]{``#1''}
\newcommand{\hs}[1]{\textsc{\hfill\scriptsize\color{dimgray}#1}}
\newcommand{\g}[1]{{\color{gray}#1}}
\newcommand{\dg}[1]{{\color{dimgray}#1}}
\newcommand{\sg}[1]{{\color{slategray}#1}}
\newcommand{\bdf}[1]{{\color{bleudefrance}#1}}
\newcommand{\itemcolor}[1]{\renewcommand{\makelabel}[1]{\color{#1}\hfil ##1}}
\newcommand\Wider[2][2em]{
\makebox[\linewidth][c]{
  \begin{minipage}{\dimexpr\textwidth+#1\relax}
  \raggedright#2
  \end{minipage}
  }
}

% misc
\linespread{1.35}

% Math stuff
\newcommand{\norm}[1]{\left\lVert#1\right\rVert}
\newcommand{\R}{\mathbb{R}}
\newcommand{\E}{\mathbb{E}}
\newcommand{\V}{\mathbb{V}}
\newcommand{\probP}{\text{I\kern-0.15em P}}
\newcommand{\ol}{\overline}
%\newcommand{\ul}{\underline}
\newcommand{\pp}{{\prime \prime}}
\newcommand{\ppp}{{\prime \prime \prime}}
\newcommand{\policy}{\gamma}
\newcommand{\plim}{ \overset{p}{\to}}
\newcommand{\hnot}{ \overset{H_0}{\to}}

% Causal Graphs
\usetikzlibrary{shapes,decorations,arrows,calc,arrows.meta,fit,positioning}
\tikzset{
    -Latex,auto,node distance =1 cm and 1 cm,semithick,
    state/.style ={ellipse, draw, minimum width = 0.7 cm},
    point/.style = {circle, draw, inner sep=0.04cm,fill,node contents={}},
    bidirected/.style={Latex-Latex,dashed},
    el/.style = {inner sep=2pt, align=left, sloped}
}

% title page
\title{Mixed Logit}
\author{Chris Conlon}
\institute{NYU Stern: MSQE Applied Econometrics}

\date{Spring 2023}




\begin{document}



%--------------
% TITLE PAGE
\begin{frame}[plain] %
\titlepage
\end{frame}


\begin{frame}{Reading}
Today's reading is Chapters 6 from:\\
\href{https://eml.berkeley.edu/books/choice2.html}{Ken Train's Discrete Choice Methods with Simulation}
\end{frame}


\begin{frame}
\frametitle{Mixed Logit}
We relax the IIA property by mixing over various logits with \alert{random effects}:
\begin{eqnarray*}
u_{ijt} &=& \beta \, x_j + \mu_{ij} + \varepsilon_{ij} \\
s_{ij} &=& \int \frac{\beta \, exp[x_{j} + \mu_{ij} ]}{1+\sum_k \exp[ \beta  \, x_{k}+ \mu_{ik} ]} f(\boldsymbol{\mu_i} | \theta)
\end{eqnarray*}
 \begin{itemize}
 \item Each individual draws a vector $\boldsymbol{\mu_i}$ of $\mu_{ij}$ (separately from $\boldsymbol{\varepsilon}$).
 \item Conditional on $\boldsymbol{\mu_i}$ each person follows an IIA logit model.
 \item However we integrate (or mix) over many such individuals giving us a \alert{mixed logit} or \alert{heirarchical model} (if you are a statistician)
 \item In practice these are not that different from linear \alert{random effects models} you have learned about previously.
 \item It helps to think about fixing $\boldsymbol{\mu_i}$ first and then integrate out over $\boldsymbol{\varepsilon_i}$
 \end{itemize}
\end{frame}

\begin{frame}{Mixed/ Random Coefficients Logit}
\small
As an alternative, we could have specified an error components structure on $\varepsilon_i$.
\begin{eqnarray*}
U_{ij} = \beta x_{ij} + \underbrace{\nu_i z_{ij} + \varepsilon_{ij}}_{\tilde{\varepsilon}_{ij}}
\end{eqnarray*}
\begin{itemize}
\item The key is that $\nu_i$ is unobserved and mean zero. But that $x_{ij},z_{ij}$ are observed per usual and $\varepsilon_{ij}$ is IID Type I EV.
\item This allows for a heteroskedastic structure on $\varepsilon_{i}$, but only one which we can project down onto the space of $z$.
\end{itemize}
An alternative is to allow for individuals to have random variation in $\beta_i$:
\begin{eqnarray*}
U_{ij} = \beta_i x_{ij} +  \varepsilon_{ij}
\end{eqnarray*}
Which is the random coefficients formulation (these are the same model).
\end{frame}

%
%\begin{frame}{Mixed/ Random Coefficients Logit}
%For each individual $i$, the resulting choice probability follows a logit:
%\begin{eqnarray*}
%P_{ij} = \int \frac{ e^{V_{ij}(\beta_i)}}{\sum_k e^{V_{ik}(\beta_i)}} f(\beta_i | \theta) \partial \beta
%\end{eqnarray*}
%This structure is quite general:
%\begin{itemize}
%\item The choice probabilities are know a function of unknown parameters $\theta$.
%\item We can allow for there to be two types of $\beta_i$ in the population (high-type, low-type). \alert{latent class model}.
%\item We can allow $\beta_i$ to follow an independent normal distribution for each component of $x_{ij}$ such as $\beta_i = \overline{\beta} + \nu_i \sigma$.
%\item We can allow for correlated normal draws using the Cholesky root of the covariance matrix.
%\item Can allow for non-normal distributions too (lognormal, exponential). Why is normal so easy?
%\end{itemize}
%\end{frame}

\begin{frame}{Mixed/ Random Coefficients Logit}
\begin{itemize}
\item Kinds of heterogeneity
\begin{itemize}
\item We can allow for there to be two types of $\beta_i$ in the population (high-type, low-type). \alert{latent class model}.
\item We can allow $\beta_i$ to follow an independent normal distribution for each component of $x_{ij}$ such as $\beta_i = \overline{\beta} + \nu_i \sigma$.
\item We can allow for correlated normal draws using the Cholesky root of the covariance matrix.
\item Can allow for non-normal distributions too (lognormal, exponential). Why is normal so easy?
\end{itemize}
\item The structure is extremely flexible but at a cost.
\item We generally must perform the integration numerically.
\item High-dimensional numerical integration is difficult. In fact, integration in dimension 8 or higher makes me very nervous.
\item We need to be parsimonious in how many variables have unobservable heterogeneity.
\item Again observed heterogeneity does not make life difficult so the more of that the better!
\end{itemize}
\end{frame}

\begin{frame}
\frametitle{Mixed Logit}
How does it work?
 \begin{itemize}
\item Well we are mixing over individuals who conditional on $\beta_i$ or $\mu_i$ follow logit substitution patterns, however they may differ wildly in their $s_{ij}$ and hence their substitution patterns.
\item For example if we are buying cameras: I may care a lot about price, you may care a lot about megapixels, and someone else may care mostly about zoom.
\item The basic idea is that we need to explain the heteroskedasticity of $Cov(\varepsilon_i, \varepsilon_j)$ what random coefficients do is let us use a basis from our $X$'s.
\item If our $X$'s are able to span the space effectively, then an RC logit model can approximate any arbitrary RUM (McFadden and Train 2002). 
\item Of course if you have 1000 products and two random coefficients, you are asking for a lot.
 \end{itemize}
\end{frame}


%\begin{frame}{Mixed/ Random Coefficients Logit}
%How do we approximate:
%\begin{eqnarray*}
%P_{ij} = \int \frac{ e^{V_{ij}(\beta_i)}}{\sum_k e^{V_{ik}(\beta_i)}} f(\beta_i | \theta) \partial \beta
%\end{eqnarray*}
%
%\begin{itemize}
%\item Monte Carlo Integration
%\begin{itemize}
%\item Draw $\beta_i$ from the candidate distribution. $[\beta_i^{(1)}, \beta_i^{(2)}, \ldots\beta_i^{(s)}] | \theta$.
%\item For each $\beta_i$ calculate $P_{ij}(\beta_i)$.
%\item $\frac{1}{S} \sum_{s=1}^S P_{ij} = \widehat{P_{j}^{s}}$
%\end{itemize}
%\end{itemize}
%The way we usually get correlated normal variables (or any normal variables) is to transform independent normals appropriately.
%\end{frame}

\begin{frame}{Mixed/ Random Coefficients Logit}
Suppose there is only one random coefficient, and the others are fixed:
\begin{itemize}
\item $f(\beta_i \theta) \sim N(\overline{\beta},\sigma)$.
\item We can re-write this as the integral over a transformed standard normal density
\begin{align*}
s_{ij}(\theta) = \int \frac{ e^{V_{ij}(\nu_{\iota},\theta)}}{\sum_k e^{V_{ik}(\nu_{\iota},\theta)}} f(\nu_{\iota}) \partial \nu
\approx \sum_{\iota=1}^{I} w_{\iota} \cdot \frac{ e^{V_{ij}(\nu_{\iota},\theta)}}{\sum_k e^{V_{ik}(\nu_{\iota},\theta)}} 
\end{align*}
\end{itemize}
\end{frame}

\begin{frame}{Numerical Integration: Monte Carlo}
How do we choose $(w_{\iota}, \nu_{\iota})$
\begin{align*}
s_{ij}(\theta)
\approx \sum_{\iota=1}^{I} w_{\iota} \cdot \frac{ e^{V_{ij}(\nu_{\iota},\theta)}}{\sum_k e^{V_{ik}(\nu_{\iota},\theta)}} 
\end{align*}
Monte Carlo Integration: Independent Normal Case
\begin{itemize}
\item Draw $\nu_i$ from the standard normal distribution.
\item Set $w_{\iota} = \frac{1}{I}$ (the number of draws).
\item Now we can rewrite $\beta_{\iota} = \overline{\beta} + \nu_{\iota} \sigma$
\item For each $\beta_{\iota}$ calculate $s_{ij}(\beta_{\iota})$.
\item $\frac{1}{I} \sum_{\iota=1}^I s_{ij}(\beta_{\iota}) = \widehat{s_{ij}}$
\end{itemize}
\end{frame}


\begin{frame}{Numerical Integration: Multivariate Normal}
Suppose instead we want to integrate out over a multivariate normal so that $\nu_i \sim \mathcal{N}(0,\Sigma)$.

\begin{itemize}
\item Work with the \alert{Cholesky Root} of $L L' = \Sigma$
\item If $\nu_i$ is a $k$-dimensional \alert{standard normal} then $L\, \nu_i \sim \mathcal{N}(0,\Sigma)$.
\item The vector $\beta_{\iota} = \overline{\beta} + L\, \nu_{\iota}$
\item Otherwise the same as before, except now we are looking for \alert{lower triangle} $L$ instead of $\sigma$.
\end{itemize}
\end{frame}



\begin{frame}{Numerical Integration: Gaussian Quadrature}
\begin{itemize}
\item Quadrature rules give us a set of $(w_{\iota},\nu_{\iota})$ to approximate
\begin{align*}
s_{ij}(\theta)
\approx \sum_{\iota=1}^{I} w_{\iota} \cdot \frac{ e^{V_{ij}(\nu_{\iota},\theta)}}{\sum_k e^{V_{ik}(\nu_{\iota},\theta)}} 
\end{align*}
to a high degree of accuracy with a small number of \alert{nodes}.
\item These points are chosen to approximate the function to a \alert{polynomial order} and then integrate the polynomial exactly.
\item ie: approximate with 13th order polynomial.
\item For smooth, bounded, and continuously differentiable functions, these work really well!
\item We will use \alert{Gauss Hermite} rules on the homework.
\end{itemize}
\end{frame}





\begin{frame}{Quadrature in higher dimensions}
\begin{itemize}
\item Quadrature is great in low dimensions -- but scales badly in high dimensions.
\item If we need $N_a$ points to accurately approximate the integral in $d=1$ then we need $N_a^d$ points in dimension $d$ (using the tensor product of quadrature rules).
\item There is some research on quadrature rules that nest and also how to carefully eliminate points so that the number doesn't grow so quickly.
\item Try \url{http://sparse-grids.de}
\end{itemize}
\end{frame}

\begin{frame}{Estimation}
How do we actually estimate these models?
\begin{itemize}
\item In practice we should be able to do MLE.
\begin{eqnarray*}
\max_{\theta} \sum_{i=1}^N y_{ij} \log s_{ij}(\theta)
\end{eqnarray*}
\item When we are doing IIA logit, this problem is globally convex and is easy to estimate using Newton's Method.
\item When doing nested logit or random coefficients logit, it generally is non-convex which can make life difficult.
\item The tough part is generally working out what $\frac{\partial \log s_{ij}}{\partial \theta}$ is, especially when we need to simulate to obtain $s_{ij}$.
\item It turns out that MSLE actually has consistent problems for fixed $S$. Why?
\item Alternative? MSM/MoM type estimators
\end{itemize}
\end{frame}




%\begin{frame}
%\frametitle{Mixed Logit}
% \begin{itemize}
%\item Now the choice of $\mu_{ij}$ is crucial.  
%\item A popular choice is \alert{random coefficients logit}: $\beta_i = \overline{\beta} +  \Sigma \nu_i * \mathbf{x_j}$ where $\nu_i$ is a vector of standard normal draws and $\Sigma$ are covariance parameters so that $\beta_i \sim N(\overline{\beta},\Sigma)$.
%\item We have to estimate $(\overline{\beta},\Sigma)$.
%\item People are often lazy and let $\Sigma$ be a diagonal matrix to reduce the \# of parameters.
% \end{itemize}
%\end{frame}


\begin{frame}
\frametitle{Mixed Logit: Estimation}
 \begin{itemize}
\item Just like before, we do MLE
\item One wrinkle--how do we compute the integral?
 \end{itemize}
\begin{eqnarray*}
s_{ij} &=& \int \frac{\exp[x_{j} \beta_i  ]}{1+\sum_k \exp[x_{k} \beta_i  ]} f(\beta_i | \theta) \approx \sum_{s=1}^{ns} w_s \frac{\exp[x_{j} (\overline{\beta} + \Sigma \nu_{is})  ]}{1+\sum_k \exp[x_{k} (\overline{\beta} + \Sigma \nu_{is})  ]} 
\end{eqnarray*}
 \begin{itemize}
\item Option 1: Monte Carlo integration.  Draw $NS=1000$ or so samples of $\nu_i$ from the standard normal and set $w_i = \frac{1}{NS}$.
\item Option 2: Quadrature. Choose $\nu_i$ and $w_i$ according to a Gaussian quadrature rule. Like \texttt{quad} in MATLAB or \texttt{mvquad} in R or \texttt{scipy.integrate.quadrature} in \texttt{Python}.
\item Personally I get nervous about integrals in dimension greater than 5. People routinely have 20 or more though.
 \end{itemize}
\end{frame}

\begin{frame}
\frametitle{Mixed Logit: Hints}
 How bad is the simulation error?
 \begin{itemize}
\item Depends how small your shares are. 
\item Since you care about $\log s_{jt}$ when shares are small, tiny errors can be enormous.
\item Often it is pretty bad.
\item I recommend sticking with quadrature at a high level of precision.
\item \texttt{sparse-grids.de} provide efficient high dimensional quadrature rules.
 \end{itemize}
\end{frame}

\section{A Semiparametric Estimator}

\begin{frame}
\frametitle{Even More Flexibility (Fox, Kim, Ryan, Bajari)}
Suppose we wanted to nonparametrically estimate $f(\beta_i | \theta)$ instead of assuming that it is normal or log-normal.
\begin{eqnarray*}
s_{ij} &=& \int \frac{\exp[x_{j} \beta_i  ]}{1+\sum_k \exp[x_{k} \beta_i  ]} f(\beta_i | \theta)
\end{eqnarray*}
\begin{itemize}
\item Choose a distribution $g(\beta_i)$ that is more spread out that $f(\beta_i | \theta)$
\item Draw several $\beta_{s}$ from that distribution (maybe 500-1000).
\item Compute $\hat{s}_{ij}(\beta_s)$ for each draw of $\beta_s$ and each $j$.
\item Holding $\hat{s}_{ij}(\beta_s)$ fixed, look for $w_s$ that solve
\begin{eqnarray*}
\min_w \left(s_j -  \sum_{s=1}^{ns} w_s \hat{s}_{ij}(\beta_s) \right)^2 \quad \mbox{ s.t. } \sum_{s=1}^{ns} w_s = 1, \quad w_s \geq 0 \quad \forall s
\end{eqnarray*}
\end{itemize}
\end{frame}

\begin{frame}
\frametitle{Even More Flexibility (Fox, Kim, Ryan, Bajari)}
% Hey this is lasso
\begin{itemize}
\item Like other semi-/non- parametric estimators, when it works it is both general and very easy.
\item We are solving a least squares problem with constraints: positive coefficients, coefficients sum to 1.
\item It tends to produce \alert{sparse models} with only a small number of $\beta_s$ getting positive weights.
\item This is way easier than solving a random coefficients logit model with all but the simplest distributions.
\item There is a bias-variance tradeoff in choosing $g(\beta_i)$.
\item Incorporating parameters that are not random coefficients loses some of the simplicity.
\item I have no idea how to do this with large numbers of fixed effects.
\end{itemize}
\end{frame}




\section*{Thanks}
\end{document}

